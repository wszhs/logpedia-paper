%%%%%%%%%%%%%%%%%%%%%%%%%%%%%%%%%%%%%%%%%%%%%%%%%%%%%%%%%%%%%%%%%%%%%%%%%%%%%%%%
% Template for USENIX papers.
%
% History:
%
% - TEMPLATE for Usenix papers, specifically to meet requirements of
%   USENIX '05. originally a template for producing IEEE-format
%   articles using LaTeX. written by Matthew Ward, CS Department,
%   Worcester Polytechnic Institute. adapted by David Beazley for his
%   excellent SWIG paper in Proceedings, Tcl 96. turned into a
%   smartass generic template by De Clarke, with thanks to both the
%   above pioneers. Use at your own risk. Complaints to /dev/null.
%   Make it two column with no page numbering, default is 10 point.
%
% - Munged by Fred Douglis <douglis@research.att.com> 10/97 to
%   separate the .sty file from the LaTeX source template, so that
%   people can more easily include the .sty file into an existing
%   document. Also changed to more closely follow the style guidelines
%   as represented by the Word sample file.
%
% - Note that since 2010, USENIX does not require endnotes. If you
%   want foot of page notes, don't include the endnotes package in the
%   usepackage command, below.
% - This version uses the latex2e styles, not the very ancient 2.09
%   stuff.
%
% - Updated July 2018: Text block size changed from 6.5" to 7"
%
% - Updated Dec 2018 for ATC'19:
%
%   * Revised text to pass HotCRP's auto-formatting check, with
%     hotcrp.settings.submission_form.body_font_size=10pt, and
%     hotcrp.settings.submission_form.line_height=12pt
%
%   * Switched from \endnote-s to \footnote-s to match Usenix's policy.
%
%   * \section* => \begin{abstract} ... \end{abstract}
%
%   * Make template self-contained in terms of bibtex entires, to allow
%     this file to be compiled. (And changing refs style to 'plain'.)
%
%   * Make template self-contained in terms of figures, to
%     allow this file to be compiled. 
%
%   * Added packages for hyperref, embedding fonts, and improving
%     appearance.
%   
%   * Removed outdated text.
%
%%%%%%%%%%%%%%%%%%%%%%%%%%%%%%%%%%%%%%%%%%%%%%%%%%%%%%%%%%%%%%%%%%%%%%%%%%%%%%%%


\documentclass[letterpaper,twocolumn,10pt]{article}
\usepackage{usenix-2020-09}

% to be able to draw some self-contained figs
\usepackage{tikz}
\usepackage{amsmath}
\usepackage{algorithm}
\usepackage{algorithmicx}
\usepackage{algpseudocode}
\usepackage{multirow}
\usepackage{amsmath}
\usepackage{times,url,color,soul,xspace,enumitem}
\usepackage{graphicx}
\usepackage{caption}
\usepackage{tabularx}
\usepackage{subcaption}
\usepackage{comment}
\usepackage{xspace}
% to be able to draw some self-contained figs
\usepackage{tikz}
\usepackage{amsmath}
\usepackage[inline,draft,nomargin,index]{fixme}

\usepackage[utf8]{inputenc}
\usepackage{amsmath, amsthm, amssymb}
\newtheorem{definition}{Definition}


\usepackage{cleveref}
\crefformat{section}{\S#2#1#3} % see manual of cleveref, section 8.2.1
\crefformat{subsection}{\S#2#1#3}
\crefformat{subsubsection}{\S#2#1#3}

\FXRegisterAuthor{giu}{agiu}{\textcolor{red}{Giu}}
\newcommand{\giu}[1]{\giunote{\textcolor{red}{#1}}}

\FXRegisterAuthor{atr}{aatr}{\textcolor{blue}{Atr}}
\newcommand{\atr}[1]{\atrnote{\textcolor{blue}{#1}}}

\FXRegisterAuthor{lin}{alin}{\textcolor{green}{Lin}}
\newcommand{\lin}[1]{\linnote{\textcolor{green}{#1}}}


 %\newcommand{\camready}[1]{\hl{#1}}
\newcommand{\camready}[1]{#1}
 %\usepackage{ulem}
 %\newcommand{\camdel}[1]{\sout{#1}}

%\newcommand{\camready}[1]{#1}
%\newcommand{\camdel}[1]{#1}

\newcommand{\dc}{datacenter\xspace}

%-------------------------------------------------------------------------------
\begin{document}
%-------------------------------------------------------------------------------

%don't want date printed
\date{}

% make title bold and 14 pt font (Latex default is non-bold, 16 pt)
\title{\Large \bf ProCon: explaining system intrusion detection with referenceable attack and common knowledge}

%for single author (just remove % characters)
\author{
{\rm Me}\\
VU, Amsterdam
\and
{\rm Other Smart People}\\
VU, Amsterdam
% copy the following lines to add more authors
% \and
% {\rm Name}\\
%Name Institution
} % end author

\maketitle
% \tableofcontents

%-------------------------------------------------------------------------------
\begin{abstract}
%-------------------------------------------------------------------------------
Advanced Persistent Threat (APT) attacks have become a pressing concern within critical sectors, including banking, military, and government, owing to their stealthy nature and prolonged presence. Traditional detection methods, such as those reliant on provenance graphs, struggle to capture subtle attack signals, primarily due to the limited information and knowledge encapsulated within log files.
Furthermore, these approaches often yield ambiguous and non-explainable results and cannot cope with the continual evolution of attack strategies. 
To address the compelling imperative for security personnel to swiftly and comprehensively detect and understand the attacks, the development of more transparent and advanced detection techniques becomes of paramount importance. 
% YD: I comment out the following sentences.
% The solution lies in technologies capable of rapidly enhancing these methods with added knowledge.
% Fortunately, with recent advancements, Large Language Models (LLMs) have emerged as particularly promising in knowledge-centric tasks. 
In this work, we propose \tool, an effective and explainable APT attack-detection method that merges the effectiveness of provenance graph-based APT detection with the knowledge acquisition capabilities of Large Language Models (LLMs).
Specifically, \tool first utilizes LLMs to extract knowledge of system processes and construct a more detailed profile for each process. Then, \tool transforms each profile into a set of key constraints, upon which \tool can perform comprehensive threat detection.
Drawing on the extensive knowledge gained through LLMs, \tool can effectively reduce ?\% false alarms and uncover ?\% more attack patterns that typically evade existing methods. This heightened capability also enables  \tool to provide deeper insights into the potential threats detected. Additionally, benefiting from its automated extraction capability, 
%which facilitates swift profile updates and eliminates the dependence on the predefined attack patterns, 
\tool has the ability to effectively and efficiently identify novel and evolving threats.
% enables swift profile updates, and without relying on predefined attack patterns, \tool can identify novel and evolving threats.
% As we spent around 21 minutes and 3.5\$ on each of 100 system profiles, \tool effectively reduces false alarms and provides deeper insight into threats, facilitating faster and more informed responses.

\end{abstract}

%-------------------------------------------------------------------------------
\section{Introduction}
%-------------------------------------------------------------------------------
Advanced Persistent Threats (APTs) have emerged as a formidable adversary in the cyber landscape. Their stealthy, persistent nature, coupled with the extensive damage they can inflict, make them exceptionally challenging to detect and mitigate. 
In light of this, we embarked on an exhaustive study of current commercial Endpoint Detection and Response (EDR) products \cite{karantzas2021empirical} and provenance graph-based solutions \cite{cheng2023kairos,alsaheel2021atlas,han2020unicorn,inam2022sok,han2021sigl}. Our deep dive into these systems allowed us to identify critical gaps and challenges. Specifically, we distilled the problems facing current solutions into four key dimensions:

\begin{itemize}
    \item \textbf{Insufficient information within Log Data}: Effective intrusion detection hinges on the richness and comprehensiveness of audit logs. Recent research \cite{gandhi2023rethinking} underscores a concerning deficiency: current logging systems capture insufficient information for reliable attack detection. Solely relying on these raw logs may result in a low attack detection coverage. To further enhance this capability, we aim to detect more attacks without increasing the granularity of log collection. A need exists to integrate more detailed knowledge into our system, especially in understanding the precise meanings of entities within logs, including process names, registry files and dynamic link libraries (DLLs), etc. By enriching our logs with more information and knowledge, we stand a better chance of detecting a wider range of attacks, especially the stealthy behaviors often employed in APT attacks.
    \item \textbf{Attack Agnosticity}: 
    Zero-day vulnerabilities (malware or flaws not yet identified by security analysts), especially in APT attacks, require detection systems that are not bound by pre-existing signatures or indicators. Anomaly detection \cite{wang2020you, alsaheel2021atlas, han2020unicorn}, stands out as the most effective technique for identifying such zero-day exploits. Therefore, an optimal method of detecting these unknown threats would not rely on known attack patterns.
    \item \textbf{Evolving Threats}: Advanced Persistent Threats (APT) represent dynamic, sustained, and sophisticated threats. Professional attacker groups continuously innovate, targeting an ever-expanding array of system processes and adapting to the latest defensive measures. This evolving landscape necessitates detection methods that can swiftly identify and adapt to emerging attack paradigms.
    \item \textbf{Transparency}: The value of detection alerts lies in their clarity. For security personnel, a clear understanding of the 'why' and 'how' behind detection can drive quicker and more effective interventions. By providing clear explanations, system administrators will have a better understanding of the threat landscape, allowing them to respond to threats faster and in a more knowledgeable manner.
\end{itemize}

While data provenance-based threat detection has emerged as a promising approach against the covert Advanced Persistent Threats (APTs), current methodologies exhibit specific shortcomings, failing to satisfy all four dimensions concurrently.
% \yd{YD: Maybe can merge the following content into above four key dimensions?}
Commercial products designed to detect APT attacks, along with \textbf{misuse-based} APT research \cite{milajerdi2019holmes,milajerdi2019poirot,hossain2020combating}, predominantly hinge on rules defined by security experts. While these misuse-oriented strategies excel at identifying known threats, they lack flexibility when confronted with novel, undefined threats. Consequently, they are constrained by predefined security policies, struggling to swiftly track evolving attacks.
On the other hand, while \textbf{anomaly-based} techniques \cite{wang2022threatrace,han2020unicorn,wang2020you} are adept at spotting deviations, their outcomes often lack transparency and pose challenges for integration into commercial products. Within this realm, graph \cite{manzoor2016fast,han2020unicorn,li2021hierarchical,yang2023prographer,cheng2023kairos} and path-based methods \cite{wang2020you,alsaheel2021atlas} tend to detect only those attacks that leave discernible traces on the graph or path, yielding results that are too macroscopic. Node-based methods provide finer granularity but can sometimes oversimplify intricate attack behaviors. 
Knowledge graph embedding techniques\cite{zeng2021watson,zengy2022shadewatcher} may have difficulty representing a single process accurately with a single vector due to the diversity of node functions and similarities between normal nodes and malicious nodes.
\textbf{Statistical-based} techniques \cite{liu2018towards,hassan2019nodoze,hassan2020we}, although intuitive, grapple with false positives, making it tough to differentiate genuinely anomalous behaviors from benign novelties. 
In sum, while each technique offers nuanced advantages, a comprehensive method that addresses all four dimensions and seamlessly tackles every facet of APT detection remains an open challenge.


The primary impediments to addressing the current challenges in APT detection can be attributed to the inherently knowledge-intensive nature of the cybersecurity domain. When faced with audit logs that present insufficient information, there's an acute need to augment them with supplementary domain-specific knowledge. This supplemental knowledge becomes even more crucial when attacks evolve or present themselves in stealthy manners. Moreover, a well-rounded domain understanding also serves as a basis for explaining detection results, enhancing transparency for security analysts. Thus, the key to overcoming these challenges lies in the ability to swiftly and effectively harness extensive cybersecurity knowledge.
Fortunately, Large Language Models (LLMs) have demonstrated remarkable prowess in recent years. These models are capable of understanding human-like text intricacies and have demonstrated efficacy across a wide range of domains, especially those requiring rapid knowledge extraction. There is an important question here: \textit{Can LLMs be effectively leveraged in cybersecurity to improve detection methods for stealthy and evolving APT attacks?}

In response to this crucial question, we focus on addressing these dimensions, combining knowledge extraction prowess of LLMs with the structural expertise of provenance graphs, aiming to detect APT. To accomplish our goal, we want the LLMs to automatically identify key constraints associated with system processes. These constraints are then transformed into actionable rules that serve as vital tools for detecting attacks. It is possible to detect attacks without increasing the granularity of our log data by thoroughly extracting and understanding log entities, such as program names, dynamic link libraries (DLLs), and registers. In addition, LLMs' vast knowledge repository allows us to discover unknown and evolving threats in addition to interpreting the nature of the attacks. As a result, we cover the four dimensions previously discussed.

As we all known, the use of LLMs for contract process profiling has both advantages and challenges. It is evident that LLMs have extensive knowledge of process behavior and are able to explain their outputs. However, following detailed practice, we found that LLMs construction processes presented challenges due to  \textit{difficulty handling complex scenarios}, 
\textit{context reliance}, \textit{memory limitations}, and \textit{the possibility of generating inaccurate information or hallucinations}.

To address these challenges, we introduced ProfileGuard, a method for automated system process profile construction using LLMs.This method involves:

\begin{itemize}
    \item Process Behavior Tree Construction: We attempted to capture as many behaviors as possible for each process, creating a behavior tree. Using a "self-ask" approach, the LLMs continuously expanded its knowledge of this tree. This tree also played a pivotal role in providing the LLMs with contextual depth.
    \item Command Execution: Based on the behavior tree, the LLMs scripted and executed commands for relevant behaviors. 
    \item Constraint Extraction: We designed a hybrid method that combines traditional programming techniques With queries to the LLMs to extract different process constraints. 
    \item Validation: A two-tiered validation approach has been implemented to validate the model's outputs. In the first tier, we execute actual commands and verify them against real-world logs, followed by a multi-round debate among the LLMs to validate their responses and reasoning.
\end{itemize}
We will delve into the mechanics of our design in detail in Section~\ref{sec:motivation}.


By using C++ and Caldera, we were able to comprehensively simulate various aspects of APT attacks. Based on existing APT attack documentation, we simulated 10 APT attack scenarios incorporating four prevalent stealth techniques, 23 malicious functionalities, and 4 prevalent stealth techniques. In the process of building 100 profiles for critical system processes, we spent approximately an hour and \$3.5 per profile. Further, we collected malware used in known APT attacks from public repositories and websites, revealing that camouflage techniques are ubiquitous in such attacks.
As a result, the detection rates and range of threats identified by ProCon were consistently superior to those of existing solutions. With our realistic and multifaceted attack simulations, ProCon's robust performance accentuates its transformative potential in cybersecurity, offering organizations a stronger defense against APT threats.



\section{Background and Related work}

\subsection{Stealthy APT Attacks}

Advanced Persistent Threats (APTs) are sophisticated, targeted, evolving, and steathy cyberattacks orchestrated by specialized groups against high-risk entities, such as nuclear power plants, banks, and governments. Amid the dynamic cyber threat landscape, APTs are employing increasingly stealthy tactics like name obfuscation, process injection, and the manipulation of legitimate system utilities to bypass traditional security defenses \cite{barr2021survivalism}. Process injection, a prevalent technique, enables malicious code to operate within the confines of another process's address space. Attackers also employ tactics like executing benign binaries through tools like rundll32.exe and launching script-driven attacks using utilities such as powershell.exe. By exploiting these widely recognized and trusted tools, adversaries can discreetly penetrate and exert control over their target systems.

\subsection{Provenance Graph-Based APT Detection}
To detect the myriad of stealthy APT attacks, provenance graph-based methods have gained significant traction in the host threat detection community. Over time, its potential for accurately identifying various host-based threats has been widely acknowledged. The primary detection techniques can be categorized into Anomaly-based, Misuse-based, and Statistics-based approaches.

\textbf{Anomaly-based Approaches}:

Anomaly-based detection methods primarily train models on benign behaviors, identifying deviations as potential cyber-attacks. While these methods can achieve impressive detection accuracy by integrating the semantics of audit records into threat analysis, the existing learning solutions often do not provide insightful or explicable results. This lack of transparency can sometimes compromise their practical utility.

\textit{Graph-based Methods:} The application of graph-based analysis for anomaly detection is computationally intensive and necessitates vast training datasets. Researchers have sought to address this by embedding provenance graphs into a vector space for ML model training. Tools like StreamSpot \cite{manzoor2016fast} and Unicorn \cite{han2020unicorn} operate by analyzing information flow graphs. While Unicorn exhibits superior performance due to its thorough graph analysis, both methods face challenges due to the constraints of graph kernel methods in identifying stealthy threats. Similarly, IPG \cite{li2021hierarchical} and ProGrapher \cite{yang2023prographer} employ a graph-level approach but grapple with issues comparable to StreamSpot and Unicorn. Notably, though many systems try to harness data provenance for threat detection, there's a palpable demand for methods that adeptly combine different aspects like scope and timeliness. Among the contenders, KAIROS \cite{cheng2023kairos} shines, offering exceptional detection and computational prowess. 

\textit{Path-based Methods:} These models work by extracting subcomponents, such as causal paths, from the provenance graph and vectorizing them, thereby tapping into existing learning methodologies. ProvDetector \cite{wang2020you} seeks to detect malware by delving into the provenance graph, converting paths in the graph into embedded forms and utilizing the Local Outlier Factor method for malware detection. However, with the diverse nature of host-based threats, relying solely on paths from the provenance graph is inadequate. On the other hand, ATLAS \cite{alsaheel2021atlas} derives both attack and non-attack sequences from graphs and uses sequence models to discern attack patterns, recognizing that distinct attacks might exhibit analogous abstract strategies, irrespective of exploited vulnerabilities and executed payloads.

\textit{Node-based Methods:}ThreaTrace \cite{wang2022threatrace} uses a GraphSAGE-based framework that learns every benign node's role in a system data provenance graph to capture stealthy abnormal behavior without prior knowledge of attack patterns. Additionally, THREAT RACE designs a multi-model framework to learn different kinds of benign nodes, which tackles the problem of data imbalance and effectively improves detection performance. 

\textit{Knowledge Graph Embedding-based Methods:} Tools like Watson\cite{zeng2021watson}and SHADEWATCHER \cite{zengy2022shadewatcher} excel at extracting overarching behaviors from detailed logs. However, these techniques come with a significant drawback: the loss of the original attack semantics.

\textbf{Misuse-based Approaches}:
Misuse-based detectors hunt down cyber threats by matching audit records against a knowledge base of security policies that describe attack semantics.
While such detection can maintain a low false-positive rate, developing security policies is time-consuming and inevitably requires domain expertise.
Holmes \cite{milajerdi2019holmes} leans towards alert generation, correlation, and scenario reconstruction, employing prior definitions of exploits in a provenance graph based on existing TTPs (Tactics, Techniques, and Procedures).
Poirot \cite{milajerdi2019poirot} focuses on correlating indicators and constructing attack graphs through cyber threat report expertise. 
Morse \cite{hossain2020combating} initializes confidentiality and integrity tags of six million system entities for tag propagation.
However, these misuse-based methods face challenges in detecting unknown threats that fall outside the established TTPs and reports.


\textbf{Statistics-based Approaches}:
Recent research suggests that security incidents in attack campaigns typically manifest as uncommon system activities \cite{liu2018towards,hassan2019nodoze,hassan2020we}. To assess the suspicion level of audit records, these studies measure them based on their historical frequency. While this approach is straightforward and can be effective, it is often plagued by a high rate of false positives. For instance, an alert might be triggered by an activity simply because it hasn't been observed before, even if it's just a benign process status retrieval. This method's primary limitation is its inability to distinguish between genuinely unusual records and new but semantically normal activities.



% \href{https://docs.google.com/document/d/1PYN3Wi3AN8LlJ7L2TqaUhpZW85__DpX2NzsL41FY3iY/edit}{paper summary}


\subsection{Large Language Models}
Large Language Models (LLMs), such as OpenAI's GPT series, have ushered in a new era in natural language understanding. These models, benefiting from extensive corpus training, possess a vast range of knowledge and reasoning capabilities, enabling them to handle diverse natural language processing tasks and adapt to varied scenarios. Notably, with a simple natural language prompt, LLMs can be task-oriented, executing designated tasks without specific retraining. Using the Transformer model, LLMs interpret input prompts and generate corresponding answers, where the multi-self-attention and feed-forward layers collaborate to interpret context and produce the outcome.

When it comes to the realm of cybersecurity, the capabilities of LLMs are gaining considerable attention. Their proven effectiveness spans areas like code analysis, vulnerability remediation, and attack detection. By harnessing their abilities in understanding, inference, and text generation, LLMs provide indispensable assistance to both computer science and cybersecurity arenas. The versatility of LLMs, combined with their human-like system interactions, makes them invaluable assets for cybersecurity endeavors. Given the extensive security knowledge required for APT attack detection, there's a strong belief that LLMs, with their superior knowledge extraction capabilities, can usher in new avenues for tackling such advanced threats






\section{Motivation}
\label{sec:motivation}

\begin{figure*}[ht]
    \centering
    \includegraphics[width=0.95\textwidth]{figs/example.pdf}
    \caption{A provenance graph that describes attack activity in the motivating example, automatically generated by ProCon. System entities are represented by nodes in the graph. Processes, file-type entities (dlls, registry, files, etc.), and sockets are represented by rectangles, ovals, and diamonds. Solid line ovals represent files, dashed line ovals represent registries, and dotted line ovals represent DLLs.A series of operations is represented by an edge, such as a read, a create, a load, etc. We have segmented the attack progression into six distinct steps using a light blue backdrop. The attack-related elements are highlighted in red, while those representing regular events and nodes are displayed in blue.}
    \label{fig-example} 
    \end{figure*}

In our analysis of APT attack reports \cite{eclecticiq2023,microsoft2023,paloaltonetworks2023}, we identify common attack steps and stealthy methods, such as Process Masquerade, Process Injection, Process Hollow, and DLL-Side Loading\cite{eclecticiq2023}. As a result of the use of four common obfuscation techniques, we were able to assemble ten scenarios for APT attacks using 23 malicious functions disguised as DLLs and program names.
One of attack scenario as motivating example is used to illustrate current detection methods' limitations as well as our approach's intuition.

\subsection{Motivating Example}
There are 6 classic steps that are followed in our simulated APT attack.
\begin{enumerate}
    \item Initial Access: The attacker first sends the victim a malicious pdf file \textit{xxx.pdf} that contains a virus. Unfortunately, the application \textit{Acrobat.exe} that reads PDF files does not protect against the malicious code that is hidden inside the document. As a result, the malicious code downloads the malicious \textit{xxxx.hta} file, and the malicious program \textit{svchost.exe} is then downloaded and executed within the directory \textit{c:/appdata/svchost.exe}. 
    \item Persistence: As an attempt to hide itself, this malicious \textit{C:/appdata/svchost.exe} opens \textit{rundll32.exe}, which then uses a DLL Side-Loading technique to load a malicious DLL \textit{shell32.dll}, which contains functionality for establishing persistence in the compromised system.
    \item Collection: Malicious \textit{C:/appdata/svchost.exe} then injects malicious \textit{advapi32.dll} processes into a benign \textit{C:/windows/system32/svchost.exe} and uses the malicious dll disguised as \textit{advapi32.dll} to gather information.
    \item Credential Access: Meanwhile, by exploiting a vulnerability, the attackers downgraded Kerberos to the more vulnerable NTLM protocol (\textit{Msv1\_0.dll}). In order to move lateral within the network, they stole credentials from the domain. 
    \item Exfiltration: A malicious attacker hollows out a portion of the memory space of the benign process svchost, fills it with malicious programs, and packages up the information into a ZIP file known as \textit{working.rar}.
    \item Lateral movement: After obtaining the credentials from step 4, the attacker executed the renamed \textit{python.exe} file, thus gaining lateral access to the network.
\end{enumerate}

Figure~\ref{fig-example} illustrates a simplified provenance graph derived from audit records. System entities are represented by nodes in the graph. Processes, file-type entities (dlls, registry, files, etc.), and sockets are represented by rectangles, ovals, and diamonds. Solid line ovals represent files, dashed line ovals represent registries, and dotted line ovals represent DLLs.
A series of operations is represented by an edge, such as a read, a create, a load, etc.


\subsection{Challenges to Existing Solutions}
Our simulated APT attack shows the following limitations of provenance-based threat detection:
\begin{itemize}
    \item \textit{Misuse-based Detection}:  A misuse-based detector\cite{milajerdi2019holmes,milajerdi2019poirot} detects cyber threats by matching audit records with security policies that describe attack semantics. The creation of these security policies is time-consuming and requires domain knowledge, even though such detection maintains a low false-positive rate. As shown in our example, a single TTP can correspond to a variety of different attacks, while "initial access" can be implemented in a variety of ways, our case utilizing \textit{Mshta.exe}. It is the responsibility of experts to cover all attack behaviors for a given TTP, but this is a time-consuming and labor-intensive process, and it cannot handle unknown or evolving threats. In addition, experts' subjective interpretations of attacks, varying proficiency levels, or even human error can affect the quality of policy formulation.
    \item \textit{Anomaly-based Detection}: Anomaly-based detection techniques detect deviations, but they rarely provide a deeper understanding of the underlying attack mechanisms. Identifying the root cause of this attack scenario can be difficult due to the deluge of records generated by this attack scenario. As an example, Unicorn\cite{han2020unicorn} may trigger alerts across a graph, but it cannot pinpoint which particular entities or patterns are triggered. As a result of their ability to mimic benign activities, they are practically imperceptible, and in this attack scenario, many disguised behaviors can be seen. Process hollow is one of the disguised behaviors used in step 5. We get rundll32 and the malicious svchost.exe using the same vector based on Shadewatcher\cite{zengy2022shadewatcher}, so we cannot detect exceptions. Furthermore, the number of anomalous entries in benign logs is very low (less than 1\% of all entries are malicious). The limited representation of stealthy threats in logs makes it difficult to train a robust and reliable model. 
    \item \textit{Statistics-based Detection}: Even though statistics-based approaches \cite{liu2018towards,hassan2019nodoze,hassan2020we}identify potential threats within graphs, they often misinterpret benign but rare threats. For instance, in our motivating example, svchost has many functions, one of which hosts the schedule service. False alarms can occur due to infrequent incidents being flagged as attacks.
\end{itemize}
All of these methods have difficulties in identifying attacks in a timely and accurate manner. Further, they do not provide sufficient granularity to clarify and explain specific attack behaviors, which complicates the identification and response process in the event of an attack.



\subsection{Intuition of Our Approach}
\label{sec:intuition}
As we mentioned previously, we thoroughly analyzed numerous APT attack 
reports\cite{eclecticiq2023,microsoft2023,paloaltonetworks2023} and discovered that attackers often employ disguise techniques to hide their malicious activities. 
The techniques include Process Masquerade, Process Injection, Process Hollow, and Direct Loading.
It is generally true that as attackers progress from straightforward masquerading techniques, like mimicking legitimate processes, to more sophisticated ones, such as DLL\-side loading, the stealthiness of their disguises tends to increase.

In spite of the masquerading method, there are inherent constraints associated with genuine program behavior, regardless of the disguised method - execution paths, parent-child relationships, permissions, as a process must execute operations, some of which must be performed in sequence, etc. It is impossible for legitimate processes to escape these constraints

In our step one of our simulated attack scenario, there were significant deviations from these constraints, While \textit{svchost.exe} is typically spawned by \textit{services.exe}, the malicious variant here is spawned by \textit{cmd.exe}. Additionally, its execution path was not what one might expect for an authentic \textit{svchost.exe}.
For more stealth attacks, like process injection as shown in step 3, a maliciously injected \textit{advapi32.dll} disrupts the expected sequence of events and violates the associated constraints. Thus, it is essential that profiles and constraints are taken into account as detection signals for regular processes in order to improve detection capabilities.





\section{Methodology of \tool}

\begin{figure*}[ht]
    \centering
      \includegraphics[width=0.95\textwidth]{figs/overview.pdf}
      \caption{An overview of \tool.}
    % \caption{The system process logs are collected and categorized using LLM as legitimate, illegitimate, and uncertain processes. Processes labeled uncertain or illegitimate are flagged for security analyst review (detailed in Section~\ref{sec:classifition}). If determined to be legitimate, LLM assists in the creation of a database of process constraints. This is achieved through behavior tree creation, command execution, constraint extraction and validation (explained in Section~\ref{sec:profile_con}). Logs that violate these constraints are indicative of potential threats (detailed in Section~\ref{sec:Threat_detection}).}
    \label{fig-framework}
    \end{figure*}

\subsection{Threat Model}\label{sec:threatModel}
In this paper, we consider APT attacks targeting critical entities that have been identified as critical infrastructure, and these attacks have the following characteristics:

\begin{itemize}
    \item \textbf{Stealthy}. The attacker employs covert tactics, concealing their malicious activities amid a substantial volume of benign background data, resulting in the victim system exhibiting behavior akin to a benign mode.
    \item \textbf{Evolving}. Professional attacker groups continually innovate, extending their targets to a broader array of system processes and adeptly adjusting to the latest defensive mechanisms to maintain a competitive edge.
    %As professional attacker groups continue to innovate, they are targeting an ever-expanding range of system processes and adapting to the latest defensive measures in order to stay ahead of the game. 
    \item \textbf{Frequent usage of zero-day exploits}. The attacker primarily relies on zero-day exploits, resulting in a lack of advance knowledge w.r.t. the specific attack patterns.
\end{itemize}
% Note that, as an alternative to sample-focused analysis, we propose to observe ubiquitous OS kernel processes. The system processes are always present and do not need to be identified before analysis, unlike suspicious samples.

\subsection{Technical Challenges}
A straightforward idea is to utilize the knowledge extraction capabilities in LLMs to facilitate the construction of more comprehensive profiles and constraints for victim system processes first, which can then be employed for conducting further thorough attack detection analysis. Recall that,
when dealing with the evolution of attacks or their stealthy tactics, supplemental knowledge can be very crucial. Moreover, in contrast to learning-based detection methods, this approach obviates the requirement for pre-defined attack patterns, which also enhances its adaptability and versatility in addressing evolving threat landscapes.
%How can we efficiently extract these constraints? It's a crucial question. Fortunately, LLMs offers significant advantages when it comes to building process profiles due to its exceptional knowledge extraction capabilities.
However, the direct employment of LLMs in our setting gives rise to the following three notable technical challenges:

% \circled{1} Complexity Handling: It might be difficult for LLMs to give comprehensive solutions when they are faced with complicated questions or scenarios.

\noindent
{\bf CH-\circled{1} Context Dependency.} %  : 
To achieve precise outcomes, LLMs commonly necessitate an enriched contextual environment. In the absence thereof, their responses may exhibit a propensity towards generality or imprecision. However, ... \yd{YD: Add statement about "hardness in our setting to get enriched context environment directly"}

\noindent
{\bf CH-\circled{2} Memory Limitations.} % \circled{2}
\yd{YD: Add statement about facts w.r.t long context in our setting.} However, owing to the constraints imposed by their token limits, LLMs encounter challenges when confronted with exceedingly lengthy texts. Moreover, LLMs exhibit a proclivity for prioritizing recent interactions, often at the expense of overlooking pertinent details embedded within the broader context of prior conversational history.
% Due to its token limit, LLMs cannot handle excessively long texts. In addition, LLMs tend to focus more on recent interactions and overlook previous details in a conversation history. 

\noindent
{\bf CH-\circled{3} Hallucination Problems.} % \circled{3} 
In spite of the remarkable advancement of LLMs in excelling across a diverse spectrum of understanding tasks, these intelligent models still show a significant drawback: the tendency to `hallucinate', i.e., they may generate misinformation and lead to unsafe behaviors.\yd{YD: Add refers about Hallucination.}
% This risk entails
In this work, such a risk entails the possibility of the model generating information for a given system process that appears plausible at first glance but lacks factual support or corresponds to non-existent realities.
% that might reveal plausible-looking yet factually unsupported or nonexistent information if the model does produce hallucinations. 

% Therefore, the key to building process profiles using LLMs is overcoming these challenges which is the central focus of this paper.

\subsection{Overview of \tool}
In this work, we propose an effective LLMs-based APT attack detection framework, named \tool, addressing all the above challenges. The overview of \tool can be found in Figure~\ref{fig-framework}. \tool comprises three main components: Process Preprocessing, Constraints Generation, and Threat Detection.
Given a set of logs collected from system processes, \tool first conducts process classification based on the name of 

As illustrated in Figure~\ref{fig-framework}, we accessed a log collection tool to collect logs from a variety of system processes that were running at different times in the system. Utilizing the powerful capabilities of LLMs, we were able to categorize these processes based on their names into three distinct categories: legitimate process names, illegitimate process names, and uncertain process names (due to the inherent incompleteness of the GPT's database). As for the latter two categories, we classified them as potentially malicious, requiring more investigation by security analysts in order to determine whether they are malicious or not. A more in-depth examination of this process is provided in Section~\ref{sec:classifition}.

As for processes that were classified with legitimate names, we used the LLMs to construct a database that outlined the normal constraints of the process.

In a previous Section~\ref{sec:intuition}, we briefly discussed the four challenges that exist when building process profiles based on the LLMs model. To address these challenges, we developed ProfileGuard, a method that harnesses LLMs for the automated creation of critical system process profiles based on the information provided by LLMs. ProfileGuard acts as a security guard for the system, using system process profiles to prevent malicious attackers.
We broke down the comprehensive profiling task into four manageable subtasks:
\begin{itemize}
    \item \textbf{Process Behavior Tree Construction}: Our goal was to capture a comprehensive range of behaviors for each process, creating a behavior tree. Through a "self-ask" approach, the LLMs continuously expanded its knowledge of this tree. LLMs can benefit greatly from this tree as it provides context for {\bf CH-\circled{1}}.
    \item \textbf{Command Execution}: A behavior tree was used to guide the LLMs to script and execute commands for behaviors that could be addressed. In addition, command execution generates richer contextual information for {\bf CH-\circled{1}}, and checks simple constraints, such as execution paths and parent-child relationships.
    \item \textbf{Constraint Extraction}:  In this step, we'll combine the LLMs and traditional programs to solve {\bf CH-\circled{2}}. Since the log size is large, asking GPT directly will easily exceed its memory limit and cause previous information to be lost. We employed common term and prefix-span sequence mining to discover deeper relationships between logs. Each constraint, which was identified by LLMs, was interpreted using its knowledge.
    \item \textbf{Validation}: LLMs must be accurate and consistent due to potential hallucination issues. We've implemented a two-tier validation approach to validate the model's outputs for {\bf CH-\circled{3}}. The first tier involves executing actual commands and cross-checking with real-world logs. Multiple LLMs engage in a multi-round debate to mutually verify their respective responses and reasoning, aiming to arrive at a unified conclusion.
\end{itemize}

Leveraging this process behavior database, we compared extracted log information against the established constraints. Any violation of these constraints is indicative of potential malicious activity.
A detailed discussion on this topic is set out in Section~\ref{sec:Threat_detection}. 







\section{Design of \tool}

\begin{figure*}[h]
    \centering
      \includegraphics[width=0.9\textwidth]{figs/prompt.pdf}
    \caption{The system is controlled by a central "controller" with memory storage and validation capabilities. Our controller deploys three key modules: 1) Process Behavior Tree Construction Module, which maps out behavioral patterns; 2) Command Execution Module, which deploys active processes; and 3) Log Constraint Extraction Module, which determines log-based requirements. These units meticulously craft program profiles using LLMs-assisted strategies and traditional methods.}
    \label{fig-framework-prompt}
    \end{figure*}

\subsection{Definitions}

% \subsubsection{System Entity and System Event}
\noindent
{\bf System Entity.} \yd{YD: STOP HERE!} In this work, we distinguish five principal entity categories: \textit{Processes}, \textit{Files}, \textit{Registry}, \textit{Dynamic Link Libraries (DLLs)}, and \textit{Network Connections}, the latter typically denoted by sockets. System entities possess unique attributes: the attributes associated with \textit{process} entities might include their Process ID (pid) or their executable paths. These entities' relationships and attributes are illustrated in Figure~\ref{fig-entity}. 

\noindent
{\bf System Event.} In this work, a system event $e$ is given by a tuple $\langle src, dst, rel, time\rangle$, where $src$ designates the source entity, constrained to only process entities, $dst$ indicates the target or destination entity, $rel$ denotes their interaction nature (e.g., writing into a file), and $time$ specifies the timestamp of the event occurrence.
We also illustrated the temporal relationship between events as $\langle e_1 \to e_2 \to e_3 \rangle$, indicating that the events occur in a logical sequence. 



\begin{figure}[ht]
    \centering
      \includegraphics[width=0.45\textwidth]{figs/entity.pdf}
    \caption{System Entity and System Event.}
    \label{fig-entity}
\end{figure}

\subsection{Process Preprocessing}
% \subsubsection{Process Monitoring}
\noindent
{\bf Process Monitoring.} 
Process monitoring is crucial: we need to gather logs from different processes in the system. In order to accomplish this, we use \textit{Windows}' robust log collection and processing tool, \textit{Sysmon}, to capture comprehensive system logs. \textit{Sysmon}'s default configuration ensures maximal log collection. 
However, due to the vast amount of logs, it becomes imperative to trim the data down. Using expert knowledge, we remove redundant and mundane events from the logs.

\noindent
{\bf Noisy Events Reduction.} 
% \subsubsection{Noisy Events Reduction}
Due to their inherent granularity and verbosity, audit logs introduce significant noise for analysts.
The omission of certain events does not impact the normal functioning of processes in behavioral cases. Integration of domain knowledge enhances our ability to reduce redundant events. 
To begin with, we eliminate events that repeat themselves frequently.
There are also actions that are universal to all processes, such as loading \textit{ntdll.dll}. We've streamlined these ubiquitous events for a more concise representation.

% \subsubsection{Process Classification}
% \label{sec:classifition}
\noindent
{\bf Process Classification.} \label{sec:classifition}
We created a prompt to query the LLMs about whether a particular program exists by name so that classification could be completed. Using the LLMs' response, we classified these processes into three categories: legitimate process names, illegitimate process names, 
and uncertain process names (due to the LLMs database's incompleteness). Many illegitimate process names are caused by name confusion, so we can change them to legitimate process names for further evaluation.
For the legitimate and confusion processes, we use the methods described in the following sections to construct profiles. In case of illegitimate or uncertain categories, security analysts should investigate further since they may contain malware.

\subsection{Behavioral Reference Construction}
\label{sec:profile_con}
Next, we present our profile building module in detail, which consists of three agent components and a central program controller.
\tool is controlled by a central program controller, which is often referred to as the "brain" of the system. In this central program controller, two components are integral: memory and validation mechanisms, which include both format and factual verification. 
Format validation of LLMs results can be solved with simple traditional programs.

In order to conduct the factual validation, multiple LLMs sessions are debating each other, as well as real-life logs being validated.
This central program controller controls three distinct agent modules: the Process Behavior Tree Construction Module, the Command Execution Module, and the Log Constraint Extraction Module. Each module manages LLMs sessions and context independently, ensuring both coherence and specialized expertise. It combines LLMs-driven processes (like behavior tree construction) with conventional computational tasks (like frequent sequence mining). A combination of traditional and LLMs-driven processes is used to create detailed process profiles under program controller guidance.

\subsubsection{Process Behavior Tree Construction}
The construction of \textit{Process Behavior Trees} is one of the most important steps in the process.
First, we define the \textit{Process Behavior Tree}.
\begin{figure}[h]
    \centering
    \begin{subfigure}{0.45\textwidth}
      \includegraphics[width=1\textwidth]{figs/tree1.pdf}
      \caption{Process Behavior Tree Representation}
    \end{subfigure}
    \begin{subfigure}{0.45\textwidth}
    \includegraphics[width=1\textwidth]{figs/tree2.pdf}
    \caption{Process Behavior Tree Representation in natural Language}
    \end{subfigure}
    \vspace{-0.05in}
    \caption{In two formats: a) visualized tree format; and b) natural language format encoded in LLMs.}
    \label{fig:behavior-tree}
    \vspace{-0.15in}
    \end{figure}


\begin{definition}[Process Behavior Tree]
A Process Behavior Tree (BT) is a tuple \((N, B)\), where:
\begin{enumerate}
    \item \(N\) is a set of nodes organized in a tree structure. Each node represents legitimate behavior of a process. A node possesses a unique identifier within it, as well as a special node, called the root, which has no parent and does not contain any children. Except for the root of the tree, all nodes have exactly one parent and a zero number of children. There can be a sub-behavior associated with each of these children nodes.
    \item \(B\) is a function that assigns to each node \(n \in N\) a set of attributes \(B(n)\). Each attribute is a pair \((b, w)\), where \(b\) is the behavior name and \(w\) is a description or attribute of that behavior. 
\end{enumerate}
\end{definition}

To generate as many behaviors as possible, we've crafted two prompts: the \textit{Initialization Behavior Tree Prompt} and the \textit{Expansion Behavior Tree Prompt} (A detailed prompt can be found in the Appendix~\ref{prompt-init-tree}). Initially, we employ the \textit{Initialization Behavior Tree Prompt} to create a foundational behavior tree. Subsequently, this tree is expanded through iterative rounds using the \textit{Expansion Behavior Tree Prompt}. In order to create a diversity and precision of behaviors, we have incorporated four strategies:
\begin{itemize}
    \item Creating an Initial Process Behavior Tree: To ensure a standardized behavior tree, we've defined a specific output format for the behavior tree.
    \item Different Expansion Techniques in Behavior Tree Prompt: The Behavior Tree Prompt uses two distinct expansion techniques - self-ask and layer-wise. While self-ask aims at a macroscopic expansion of behaviors, layer-wise expansion delves deeper to uncover finer, detailed behaviors. To extract a comprehensive set of behaviors, these two methods combine breadth and depth searches.
    \item Using system role: LLMs often ignore earlier details in a conversation, focusing on recent interactions. As a countermeasure, we leverage the benefits of the system role and embed both roles and objectives within it.
    \item Self-Evaluation by LLMs: We aim to provide LLMs with the ability to self-criticize and review current nodes. If the LLMs determines that the branch does not require expansion, node expansion is stopped.
\end{itemize}

The resulting behavior tree is illustrated in the Figure~\ref{fig:behavior-tree}. Taking the behavior tree of \textit{svchost.exe} as an example, we begin with the basic profile. This includes details such as the execution path of the process, its parent process (the parent process of \textit{svchost.exe} can only be \textit{services.exe}), and its child processes. Following this, we have the fundamental behaviors, such as service hosting, DLL loading, and service isolation exhibited by \textit{svchost.exe}. Focusing on the most significant behavior, which is service hosting, the behavior tree can be extended to detail the specific services being hosted. From the Figure~\ref{fig:behavior-tree}, one can observe the graphical representation of the behavior tree and its expression in natural language.


\subsubsection{Command Execution}

An initial step is to design a prompt (detailed prompts can be found in Appendix:\ref{prompt-commands}) for generating commands from previous behavior trees. In order to obtain actual log files, this prompt generates corresponding system commands. We can validate some of the behaviors described in the previous step by collecting these logs. Using these logs, it is possible to directly verify some behaviors like parent processes, execution paths, and other details. It is, however, not possible to translate all system processes and behaviors into executable commands. In these cases, we employ the LLMs to produce relevant recommendations, which guide us in manually engaging with the system to obtain the required logs.

\subsubsection{Behavioral Invariants Extraction}

\begin{figure*}[h]
  \begin{subfigure}{.5\textwidth}
      \includegraphics[width=\textwidth]{figs/svchost_constraints.pdf}
      \caption{Behaviorial Invariants of Svchost}
      \label{fig:cons-svchost}
  \end{subfigure}
  \hfill
  \begin{subfigure}{.5\textwidth}
      \includegraphics[width=\textwidth]{figs/smss_lsass_constraints.pdf}
      \caption{Behaviorial Invariants of Smss and Lsass}
      \label{fig:cons-smss-lsass}
  \end{subfigure}

  \begin{subfigure}{.5\textwidth}
      \includegraphics[width=\textwidth]{figs/svchost.pdf}
      \caption{Svchost Behaviorial Invariants in Graph format}
      \label{fig:con-svchost-tree}
  \end{subfigure}
  \hfill
  \begin{subfigure}{.5\textwidth}
      \includegraphics[width=\textwidth]{figs/lsass.pdf}
      \caption{Lsass Behaviorial Invariants in Graph format}
      \label{fig:con-lsass}
  \end{subfigure}
  \caption{Constraint Definition}
  \label{fig:cons-def}
 \end{figure*}


The extraction of system constraints is a pivotal step in our methodology. We begin by categorizing constraints into five types as shown in Figure~\ref{fig:cons-def}.
The Figure~\ref{fig:cons-def} illustrates the constraints of three processes: \textit{svchost.exe}, \textit{lsass.exe}, and \textit{smss.exe}. In Figure~\ref{fig:cons-svchost}, the constraints of \textit{svchost.exe} are shown. There are five types of constraints: execution path, parent process, child process, intrinsic constraints (\textit{svchost.exe} will always load \textit{kernel32.dll} and \textit{advapi32.dll}.), and temporal constraints (\textit{advapi32.dll}, \textit{msvct.dll}, and \textit{kernelbase.dll} must be loaded in the specified sequence).
For certain processes, like \textit{smss.exe}, parent and child processes are optional. The parent process of \textit{smss.exe} can be \textit{system} or \textit{Ntoskml.exe}.
In summary, these constraint relationships can be fully characterized using the AND, OR, and ORDER operators.

By comparing them with the logs gathered in the previous steps, we can validate and extract the first three constraints directly.
However, for intrinsic and temporal constraints, the challenge arises due to real-world logs' vastness. It is impractical to query the LLMs for each log due to memory constraints and its propensity to forget extended conversations. To overcome this, we designed a hybrid method that combines traditional programming techniques with queries to the LLMs to extract these two constraints.

To clearly describe the common items and frequent sequence mining algorithms, we first provide some definitions.
Given a set of log sequences \( \mathcal{D} \), each sequence \( S \in \mathcal{D} \) contains logs \( L \), where each log is a tuple \( L = (s, o, d,t) \) consisting of:
\begin{align*}
    s & : \text{Source process} \\
    o & : \text{Operation} \\
    d & : \text{Destination or object of the operation}\\
    t & : \text{time specifies the timestamp of the event occurrence}
\end{align*}

We want to identify behaviors that are certain to occur in a specific process and in a certain order. 
We begin by mining the logs for common items among different log sequences.   
To speed up common sequence mining, we removed content from sequences that do not include common items when getting common items.
our next step is to use the PrefixSpan algorithm to find common sequences in these logs. 
To obtain the common sequences, rather than frequent sequences, we set the threshold \( \theta =1\) of the PrefixSpan algorithm, indicating that the sequence is certain to occur in the specified order.
(Algorithms are detailed in the Appendix~\ref{alg:fre-common}).
Having established the common items and sequences, we then direct our queries towards the LLMs, focusing specifically on these elements to extract intrinsic and temporal constraints. 
The detailed set of prompts used in this process as shown in Appendix~\ref{prompt-cons-explain}.


\subsection{Runtime Behavioral Validation}

Furthermore, we request the LLMs to explain its findings, providing insights into these constraints.
Based on the prompt as shown in Appendix~\ref{prompt-cons-explain}, we ask LLMs to explain why this constraint exists, such as why \textit{lsass.exe} must load \textit{lsasrv.dll}, and why \textit{lsass.exe} must load \textit{samsrv.dll} after loading \textit{lsasrv.dll}. 
The LLMs would respond: \textit{lsass.exe} loads \textit{lsasrv.dll} primarily to utilize its code to complete tasks such as authentication and the generation of security tokens. 
\textit{lsass.exe} loads \textit{samsrv.dll} to manage and access the security account database, supporting user authentication and the implementation of local security policies

In addition, due to the potential for hallucinations or misleading results, it is imperative to ensure the accuracy and consistency of LLMs outputs. Due to this challenge, we developed a two-dimensional validation system.

\textbf{Factual Validation.}
A more important validation is the factual validation, which is designed to ensure that LLM's final output is accurate and consistent.

\begin{itemize}
    \item Real-world Cross-referencing: LLMs outputs are executed as real-world commands, and their results are cross-checked against real-world logs.
    \item LLMs Multi-session Debates: The LLMs engages in iterative debates in multiple instances. Their goal is to find a consensus on an answer that is both accurate and reliable by verifying each other's responses and reasoning. They cannot reach a consensus on actions that are not entirely correct by having multiple LLMs debate each other. For definitive behaviors, they can eventually come to an agreement. A more detailed description of our approach can be found in the Appendix~\ref{prompt-cross-validation}.
\end{itemize}



\subsection{Threat Detection}
\label{sec:Threat_detection}
An anomaly detection process is based on a multitude of rules derived from the preceding steps. In various attacks, these rules are specifically designed to pinpoint the exact constraints that have been violated. Process masquerading, for example, violates execution pathway and parent-child constraints, whereas process injection violates inherent and temporal constraints more often. This is the basis of our offline process for building individual process profiles. But during online detection, it's possible that some processes might not be on our list.
Our established methodology allows us to profile processes that exist but are not in our knowledge base.

\textbf{Construction Unseen Process Profile.}
The risks and uncertainties associated with incomplete process profiles should be managed and mitigated. Online methods can detect and prevent some attacks caused by incomplete profiles. By using this method, false alarms may be reduced for benign interactions between system entities that were not observed during the construction of the profile.
\section{Experiment Evaluation}

To validate the effectiveness of our approach, we simulated various attacks on 100 critical system processes, resulting in 10 APT (Advanced Persistent Threat) attack scenarios. We then evaluated our method based on the following five aspects:

\begin{itemize}
    \item \textbf{Q1.} How well does our method work, and does it achieve a low false-positive rate? ((§~\ref{sec-effective})
    \item \textbf{Q2.} What is the time it takes for our method to construct a profile for each process? What is the most time-consuming step in the three-step process profile creation? (§~\ref{sec-eff})
    \item \textbf{Q3.} How does the behavior diverge based on LLM, and how does the verification step optimize the outcome? (§~\ref{sec-ab-study})
    \item \textbf{Q4.} How can we validate the accuracy of these explanations using our method? (§~\ref{sec-explanation-val})
    \item \textbf{Q5.} Is it common for real-world APT attacks to disguise their processes in various ways? (§~\ref{sec-real-world})
\end{itemize}
All experiments are performed on a server with Intel Xeon E5-2620 v4 CPUs @ 2.10GHz, 64 GB physical memory, and an NVIDIA Tesla V100 GPU. The OS is Ubuntu 16.04.3 LTS.


\subsection{Implementation}
We present important technical details in the implementation.

\subsubsection{System Auditing Collection}

While ProCon is engineered to handle inputs from both Linux and Windows systems, our evaluation predominantly emphasizes Windows events. This is largely due to the fact that our benign deployment environment is majorly comprised of Windows-based hosts, and a significant portion of sophisticated malware is specifically crafted for Windows platforms. To effectively gather provenance data from these systems, we turn to Sysmon, Windows' sophisticated log collection tool. By leveraging Sysmon's default settings, we ensure a thorough and exhaustive capture of system logs.

\subsubsection{Simulated Datasets}

% Apt攻击场景
% 常见的6种伪装的类型
% Process Masquerade
% Parent PID Spoofing
% Process Hollow
% Process Injection
% DLL Side-Loading
% Functional Masquerade 

% 不同节点类型的伪装
% 文件类型
% Dll

In forensic analysis, the lack of publicly available attack datasets and system logs is a common challenge. For example, the data released by DARPA's Transparent Computing program do not include audit logs generated during evaluation engagements and also lack explicit labels. 

To tackle the challenges, we approached the problem from three perspectives: enhancing stealthy attack coverage, simulating realistic APT scenarios, and accounting for previously unseen attack methodologies. 
For our simulations, we relied on well-known hacker utilities like Caldera, complemented by our bespoke C++ programs crafted to execute a spectrum of attack functionalities. This included strategies for Initial Access, Privilege Escalation, Information Gathering, Defense Evasion, and ensuring Persistence.


\textbf{Attack Datasets.}
\begin{itemize}
    \item  Stealthy attack coverage: We rank attacks into six categories based on their level of stealth. These categories, from the least to the most stealthy, are: Process Masquerade, Parent PID Spoofing, Process Hollow, Process Injection, DLL Side-Loading, and Functional Masquerade. For basic disguises like Process Masquerade, we've created our proprietary software designed for actions like maintaining persistence and gathering information. For more sophisticated techniques such as Process Injection and DLL Side-Loading, we design malicious DLLs to execute these actions.
    \item Realistic APT scenarios: To enhance the realism of our simulations, we crafted ten APT attack scenarios based on in-depth reports of real-world Advanced Persistent Threat (APT) activities, generating audit logs within a controlled testbed environment. The process of implementing these APT attacks was as follows: Drawing from the genuine APT reports, we selected the aforementioned six stealthy attack techniques. For each technique, based on the specific details outlined in the APT reports, we designed varying payloads (either in the form of DLLs or EXEs). Each payload was tailored with distinct malicious functionalities, such as information gathering or privilege escalation, to emulate the multifaceted nature of actual APT attacks.
    \item Unseen attack methodologies: To demonstrate that our attacks can detect unknown threats, we implemented a more stealth attack known as Functional Masquerade. This involves exploiting the normal functionalities of legitimate programs for malicious activities.
\end{itemize}

\textbf{Normal Datasets.}
Additionally, similar to previous work that constructs benign system events, we emulate diverse normal user activities on the same machine during each attack execution to the best of our ability. More specifically, we manually generate various benign user activities like browsing different websites, executing different applications (e.g., reading emails, downloading attachments), and connecting to other hosts.

\textbf{Label.}
With our full knowledge of attack workflow, we manually label the ground truth of interactions through their relations to attacks.


\subsubsection{Process Classification}

In order to construct a comprehensive analysis framework, we strategically selected 100 pivotal processes for examination. The selection criteria encompassed multiple dimensions, including: 1) Core system processes that are integral to system functionality; 2) Processes highly associated with security mechanisms; and 3) Processes commonly utilized by system administrators. Additionally, we integrated an assessment of processes that are frequently targeted in cyber-attacks, based on statistical analysis derived from MITRE's official website as shown in Figure~\ref{fig-process}

\begin{figure}[h]
    \centering
      \includegraphics[width=0.49\textwidth]{figs/process.png}
    \caption{Comparison of attack frequencies for different processes based on the official MITRE website.}
    \label{fig-process}
\end{figure}

% \begin{table*}[h!]
%     \centering
%     \begin{tabularx}{\textwidth}{|l|l|X|}
%     \hline
%     \textbf{Process} & \textbf{Categories} & \textbf{Description} \\
%     \hline
%     svchost.exe & wlidsvc & Microsoft Account Sign-in Assistant which enables user to sign-in through Microsoft account identity services. \\
%     \cline{2-3}
%     & WaaSMedicSvc & Windows Update Medic Service fixes any damages suffered by the Windows Update component. \\
%     \cline{2-3}
%     & wuauserv & Windows Update Service enables the detection, download, and installation of updates for Windows. \\
%     \cline{2-3}
%     & ClipSVC & Client License Service provides infrastructure support for the Microsoft Store. \\
%     \cline{2-3}
%     & BITS & Background Intelligent Transfer Service downloads files from or uploads files to HTTP web servers and SMB file shares. \\
%     \hline
%     rundll32.exe & DispatchAPICall & Call an API function within a DLL file mainly used for executing system commands. \\
%     \cline{2-3}
%     & ShellRefresh & Refresh the Windows Universal App Package cache to resolve issues related to the installation, update, or removal of Universal Windows Platform (UWP) apps. \\
%     \cline{2-3}
%      & StateRepositoryDoMaintenanceTasks & Perform maintenance tasks on the Windows state repository to resolve issues related to the state repository. \\
%     \cline{2-3}
%     & PcaPatchSdbTask & Performing patching or updating system software or drivers \\
%     \hline
%     \end{tabularx}
%     \caption{Processes and Descriptions}
%     \label{tab:process}
% \end{table*}


\begin{table}[h]
\centering
\begin{tabular}{|c|c|c|c}
\hline
 & Precision & Recall & F1-Score \\
\hline
Process Masquerade & & &  \\
\hline
Parent PID Spoofing & & &  \\
\hline
Process Hollow & & &  \\
\hline
Dll Process Injection & & &  \\
\hline
DLL Side-Loading & & &  \\
\hline
Functional Masquerade  & & &  \\
\hline
\end{tabular}
\caption{Your table caption here}
\label{tab:my_label}
\end{table}

Descriptions of various processes can be found in the Appendix. Here, we highlight the two most common processes: svchost.exe and rundll32.exe as shown in Table~\ref{tab:process}.

\textbf{Svchost.exe processes}. svchost.exe is a Windows system program that hosts wide range of hardware and OS related services daemons in the background whose behaviors are determined by command line arguments, which load different library (e.g., .dll) files. Due to the svchost.exe’s persistent presence and its well-trusted nature, adversaries often impersonate svchost.exe to launch their attack payloads. Therefore, in addition to the benign program classification task using svchost.exe, we also exercised an anomaly detection task to distinguish between benign and malicious svchost.exe instances. Detailed list of svchost.exe programs invoked with different command line arguments are listed in Table 4 alongside their brief descriptions.

\textbf{Rundll32.exe processes}. rundll32.exe hosts specific functions exported from library (e.g., .dll) files. While similar to svchost.exe, rundll32.exe is more easily exploited, as an attacker can write their own .dll to be loaded by rundll32.exe. Security incident reports and research rank rundll32.exe as one of the most favored targets for malware impersonation because it can be used to embed malicious .dlls into the target’s address space using a legitimate host program. We conducted two sets of evaluations for rundll32.exe: program classification for benign rundll32.exe applications and anomaly detection for malware samples that impersonate rundll32.exe. The different categories of tasks include calling API functions within a .dll to execute system commands, refreshing Windows Universal App Package cache, performing maintenance tasks on the Windows state repository, and updating and patching system software or driver. These tasks can help resolve various issues related to system settings, installation and removal of UWP apps, and performance optimization. A detailed list of rundll32.exe programs that load different .dll are listed in Table 4 alongside their brief descriptions.


\subsection{Effectiveness}
\label{sec-effective}
In this study, we assess the detection efficacy of ProCon through the use of precision, recall, F1-score, and accuracy metrics. More specifically, precision measures correctly detected threats against predicted threats; recall measures correctly detected threats against ground-truth threats; and F1score calculates the harmonic mean of the precision and recall.

\subsubsection{Evaluation on Attack for Various Processes}
We initiate stealthy attacks on 100 processes and collect the attack logs. How does our method fare in terms of false negatives? We'll use a case study to analyze how our method detects malicious activities.

As mentioned in the previous section, we evaluate our approach from three distinct dimensions, with one primary focus on the coverage of attack detection. For this assessment, we systematically executed six types of stealthy attacks on 100 commonly encountered processes. Each attack payload was designed to implement one of six functionalities, such as information gathering and privilege escalation. This resulted in a comprehensive testing scenario involving 3,600 attack detection tasks. The outcomes of our method's detection are depicted in the following figure.

Subsequently, we delve into a detailed analysis to identify the reasons behind any false positives and false negatives produced by our approach.

\textbf{Analysis of False Positives.}

\textbf{Analysis of False Negatives.}

\subsubsection{Evaluation on Normal Workloads}
In 100 processes, we stimulate benign behaviors triggered by LLM and collect the corresponding logs. How does our method perform in terms of false positives? some case study will help understand why our method can achieve low false positives.

\subsubsection{Evaluation on APT scenarios}
we construct 10 APT attack scenarios by combining single attack behavior and compare our method against four state-of-the-art techniques. By analyzing cases where current methods exhibit false positives, we aim to explain why these methods might produce false positives.

\textbf{ProvDetector.}
Among the methods most analogous to ours is "ProvDetector." 
This approach \cite{wang2020you} aims to identify malware by analyzing the provenance graph. It transforms paths within the graph into embedded representations and then employs the Local Outlier Factor technique for malware detection.
We drew insights from results based on event investigation to evaluate the efficacy of our method against ProvDetector.

\textbf{ThreaTrace.}
ThreaTrace \cite{wang2022threatrace} constructs a unique model for each node type within a provenance graph, aiming to identify anomalies at the node level. Consequently, our comparison with ThreaTrace centers around the results derived from entity (or node) based detection.

\textbf{ATLAS.}
ATLAS \cite{alsaheel2021atlas} derives both attack and non-attack sequences from
graphs and uses sequence models to discern attack patterns,
recognizing that distinct attacks might exhibit analogous ab-
stract strategies, irrespective of exploited vulnerabilities and
executed payloads
We will juxtapose our results, both from the perspectives of entity (nodes) and event (edges), against those produced by ATLAS.

\textbf{Shadewatcher.}
SHADEWATCHER \cite{zengy2022shadewatcher} utilizes knowledge graph embeddings to represent each node and edge. Originally tailored for information flow, we've adapted the method to suit our comparative analysis. Focusing on process nodes, we center our representation on a process and its 1st to 2nd order neighbors, forming a subgraph. The rest of the components remain consistent with the original methodology.


\begin{table*}[htbp]
\centering
\caption{Your caption here}
\begin{tabularx}{\textwidth}{|c|c|c|c|c|}
\hline
\textbf{Attack ID} & \textbf{\#Attack Entity} & \textbf{\#Non-attack Entity} & \textbf{\#Attack Event} & \textbf{\#Non-attack Event} \\
\hline
S-1 & 22 & 7,445 & 12 & 34,008  \\
S-2 & ... & ... & ... & ...  \\
... & ... & ... & ... & ...  \\
M-6 & 42 & 19,580 & 4,598 & 90,467  \\
\hline
Avg. & 28 & 20,051 & 17,341 & 275,796 \\
\hline
\end{tabularx}
\end{table*}


\subsection{Efficiency}
\label{sec-eff}
What is the time it takes for our method to construct a profile for each process? What is the most time-consuming step in the three-step process profile creation?

\subsection{Ablation Study}
\label{sec-ab-study}

What is the time it takes for our method to construct a profile for each process? What is the most time-consuming step in the three-step process profile creation?
We conduct an ablation study to understand the contributions of the three key modules: Process Tree Construction Module, Command Execution Module, and Constraint Extraction Module, to the overall performance of ProCon GPT. We present three distinct variants for our analysis:

\begin{itemize}
    \item ProCon GPT-NO-tree-construction: In this variant, the Process Tree Construction Module is deactivated, resulting in all data being fed directly into the system without any hierarchical organization.
    \item Procon GPT-NO-Verification: The Constraint Extraction Module is disabled in this variant. Instead of using the Process Tree Construction approach, this model leverages the same techniques as those employed with LLMs for penetration testing, as outlined in the Exploratory Study.
\end{itemize}

\subsection{Explanation Validation}
\label{sec-explanation-val}
How can we validate the accuracy of these explanations using our method?
Our approach seamlessly integrates detection and interpretation. In the following sections, we will delve into how our method elucidates attack behaviors, thereby assisting security analysts in rapidly verifying and responding to threats.

\subsection{Real-world Validation}
\label{sec-real-world}
Is it common for real-world APT attacks to disguise their processes in various ways?



\section{Discussion}

% 无法检测离地攻击
% 有些进程在llm知识库中没有

\section{Conclusion}


%-------------------------------------------------------------------------------
\section*{Acknowledgments}
%-------------------------------------------------------------------------------

The USENIX latex style is old and very tired, which is why
there's no \textbackslash{}acks command for you to use when
acknowledging. Sorry.

%-------------------------------------------------------------------------------
\section*{Availability}
%-------------------------------------------------------------------------------

USENIX program committees give extra points to submissions that are
backed by artifacts that are publicly available. If you made your code
or data available, it's worth mentioning this fact in a dedicated
section.

%-------------------------------------------------------------------------------
\bibliographystyle{plain}
%\bibliography{\jobname}
\bibliography{main}

\appendix
\section{Appendix}

% \label{alg:fre-common}

\subsection{test}

\subsection{Algorithm}

\begin{algorithm}
\caption{Common Items and SubSequences Mining}
\label{alg:fre-common}
\begin{algorithmic}[1]
\Require A set of log sequences \( \mathcal{D} \), threshold \( \theta=1 \)
\Ensure Common subsequences \( \mathcal{F} \)
\Statex
\Function{extractCommonItems}{$\mathcal{D}$}
    \State \( \text{items\_set} = \) set of items from the first sequence in \( \mathcal{D} \)
    \For{each sequence \(S\) in \(\mathcal{D}\) excluding the first one}
        \State \( \text{items\_set} = \text{items\_set} \cap \) set of items from \(S\)
    \EndFor
    \State \Return \( \text{items\_set} \) 
\EndFunction
\Statex
\Function{PrefixSpanMining}{$\mathcal{D}$, $\theta$}
    \State \( \text{Implement PrefixSpan or use an existing library} \)
    \State \( \mathcal{F} = \) sequences with frequency \( \geq \theta \)
    \State \Return \( \mathcal{F} \)
\EndFunction
\Statex
\Function{Mining}{$\mathcal{D}$, $\theta$}
    \State \( \mathcal{C} \) = \Call{extractCommonItems}{$\mathcal{D}$}
    \For{each sequence \(S\) in \(\mathcal{D}\)}
        \State Remove items from \(S\) not in \( \mathcal{C} \)
    \EndFor
    \State \( \mathcal{F} \) = \Call{PrefixSpanMining}{$\mathcal{D}$, $\theta$}
    \State \Return \( \mathcal{F} \)
\EndFunction
\end{algorithmic}
\end{algorithm}


% \begin{algorithm}
% \caption{Frequent Subsequence Mining with Common Item Extraction from Log Sequences}
% \label{alg:fre-common}
% \begin{algorithmic}[1]
% \Require A set of log sequences \( \mathcal{D} \), threshold \( \theta \)
% \Ensure Frequent subsequences \( \mathcal{F} \)
% \Statex
% \Function{extractCommonItems}{$\mathcal{D}$}
%     \State $commonItems \gets \emptyset$
%     \For{each log $L$ in the first sequence of \( \mathcal{D} \)}
%         \State $isCommon \gets true$
%         \For{each $S$ in \( \mathcal{D} \)}
%             \If{log is not a subsequence of $S$}
%                 \State $isCommon \gets false$
%                 \State break
%             \EndIf
%         \EndFor
%         \If{$isCommon$}
%             \State $commonItems$.add($L$)
%         \EndIf
%     \EndFor
%     \State \Return $commonItems$
% \EndFunction
% \Statex
% \State \( \mathcal{C} \) = \Call{extractCommonItems}{$\mathcal{D}$}
% \Statex
% \Function{PrefixSpanMining}{$\mathcal{D}$, $\theta$}
%     \State $\mathcal{F} \gets \emptyset$
%     \For{each $S$ in $\mathcal{D}$}
%         \For{each log $L$ in $S$}
%             \If{\Call{support}{$L$, $\mathcal{D}$} $\geq \theta$}
%                 \State $\mathcal{F}$.add($L$)
%                 \State $projectedDB \gets$ \Call{project}{$\mathcal{D}$, $L$}
%                 \State $\mathcal{F} \gets \mathcal{F} \cup$ \Call{PrefixSpanMining}{$projectedDB$, $\theta$}
%             \EndIf
%         \EndFor
%     \EndFor
%     \State \Return $\mathcal{F}$
% \EndFunction
% \Statex
% \Function{support}{log, $\mathcal{D}$}
%     \State $count \gets 0$
%     \For{each $S$ in $\mathcal{D}$}
%         \If{log is a subsequence of $S$}
%             \State $count \gets count + 1$
%         \EndIf
%     \EndFor
%     \State \Return $count$
% \EndFunction
% \Statex
% \Function{project}{$\mathcal{D}$, log}
%     \State $projectedDB \gets \emptyset$
%     \For{each $S$ in $\mathcal{D}$}
%         \If{log is a subsequence of $S$}
%             \State append the part of $S$ after log to $projectedDB$
%         \EndIf
%     \EndFor
%     \State \Return $projectedDB$
% \EndFunction
% \end{algorithmic}
% \end{algorithm}

\begin{table*}
    \centering
    \begin{tabular}{|l|l|l|p{6cm}|}
        \hline
        Masquerading Type & Malware/APT & Targeting process & Description \\
        \hline
        \multirow{2}{*}{Process Masquerading}
        & GALLIUM\cite{cybereason2023} & \textit{cmd.exe} & Attackers use a renamed \textit{cmd.exe} file to evade detection \\
        \cline{2-4}
        & DCSrv\cite{checkpoint2021} & \textit{svchost.exe} & Attackers masquerade as a legitimate \textit{svchost.exe} process to encrypt all computer volumes \\
        \cline{2-4}
        & Egregor\cite{intrinsec} & \textit{svchost.exe} & Attackerd masquerade \textit{svchost.exe} to launch an RClone client to exfiltrate data \\
        \cline{2-4}
        & APT29\cite{mitre_g0016} & \textit{svchost.exe/rar.exe} & Renaming these benign processes for use in various stages of the attack \\
        \hline
        \multirow{3}{*}{Dynamic-link Library Injection} & Aria-body\cite{checkpoint2020} & \textit{rundll32.exe/dllhost.exe} & Aria-body loader inject itself to another process \\
        \cline{2-4}
        & BlackEnergy\cite{fsecure2019} & \textit{svchost.exe} & The driver component injects the main DLL component into \textit{svchost.exe} \\
        \cline{2-4}
        & Sykipot\cite{att2023} & \textit{outlook/iexplore.exe} & The original malware scans the list of running processes looking for outlook \\
        \hline
        \multirow{2}{*}{Process Hollowing} & RCSession\cite{secureworks} & \textit{svchost.exe} & RCSsession is extracted from a file called \textit{English.rtf} and is loaded into \textit{svchost.exe} \\
        \cline{2-4}
        & Smoke Loader\cite{malwarebytes2016} & \textit{explorer.exe} & Smoke Loader injects itself into \textit{explorer.exe} and deletes the original executable \\
        \hline
        \multirow{2}{*}{Dll side-Loading} & APT3\cite{mitre_g0022} & \textit{Chrome.exe} & side load DLLs with a valid version of Chrome with one of their tools \\
        \cline{2-4}
        & APT29\cite{mitre_g0016} & \textit{Msoev.exe} & The HTA file will drop the three executables into the \%TEMP\% directory \\
        \hline
    \end{tabular}
    \caption{Real world APT Attack using steathy technical}
    \label{tab:real_world}
\end{table*}




\subsection{Prompt}


\subsubsection{Initialization Behavior Tree Prompt}
\label{prompt-init-tree}
\begin{tabularx}{\textwidth}{|c|X|}
\hline
\multicolumn{2}{|c|}{\textbf{Input:} \colorbox{codegreen}{\{process\_name\}}} \\
\multicolumn{2}{|c|}{\textbf{Output:} \colorbox{SoftRed}{\{behavior\_tree\}}for\colorbox{codegreen}{\{process\_name\}}} \\
\hline
\textbf{Role} & You are a security expert well-versed in the legitimate behavior of crucial Windows processes. Now, you are qualified to assist in constructing behavior trees for legal processes. \\
\hline
\textbf{Goal} & 
Based on your domain expertise, create an initial \colorbox{SoftRed}{\{behavior\_tree\}} for \colorbox{codegreen}{\{process\_name\}}. 
Please note that this is an introductory prompt; focus only on primary behaviors (e.g., 1.1, 2.1) without delving into sub-layers. And concentrate on major behaviors specific to the process.
\\ 
\hline
\multirow{6}{*}{\textbf{Output Format}} & This is just a format, in reality there can be deeper and more numerous sub-behaviors: \\
& \colorbox{lightgray}{1. basic profile} \\
& 1.1. execution path \\
& 1.2. parent and child processes \\
& 1.3. permissions \\
& \colorbox{lightgray}{2. Fundamental Behavior} \\
& 2.1. xxxxx \\
& 2.2. xxxxxx \\
& 2.3. xxxxx \\
& 2.3.1 xxxxx \\
& 2.3.1.1 Some deeper behavior \\

& \dots \\
\hline
\multirow{5}{*}{\textbf{Notes}} 
& 1. Each token has a cost, so keep your output concise and avoid irrelevant information. \\
& 2. Continuously review and analyze your actions to ensure you are performing to the best of your abilities. \\
& 3. Constructively self-criticize your big-picture behavior constantly. \\
& 4. Ideally, behaviors should be generated that can be confirmed through system commands. \\
\hline
\textbf{\colorbox{codeorange}{LLM Response}} \\
& \colorbox{lightgray}{1. basic profile} \\
& 1.1. execution path: \colorbox{codeorange}{C:/Windows/System32/svchost.exe} \\
& 1.2. Parent and Child Processes: Parent process is usually \colorbox{codeorange}{services.exe}. Child processes can vary based on the specific service \\
& 1.3. Permissions: svchost.exe typically runs with system-level permissions \\
& \colorbox{lightgray}{2. Fundamental Behavior} \\
& 2.1. Service Hosting: Hosts multiple Windows operating system services \\
& 2.2. Dynamic Link Library Loading: Loads DLL files that contain the actual code for the service \\
& 2.3. Network Communication: Depending on the service, it may communicate over the network \\
& 2.4. System Resource Management: Manages system resources such as memory and CPU for the services it hosts. \\
\\
\hline
\end{tabularx}

\clearpage
\subsubsection{Expansion Behavior Tree Prompt}
\label{prompt-expan-tree}

\begin{tabularx}{\textwidth}{|c|X|}
\hline
\multicolumn{2}{|c|}{\textbf{Input:} \colorbox{SoftRed}{\{behavior\_tree\}}for\colorbox{codegreen}{\{process\_name\}}} \\
\multicolumn{2}{|c|}{\textbf{Output:} \colorbox{SoftRed}{\{behavior\_tree\}}for\colorbox{codegreen}{\{process\_name\}}} \\
\hline
\textbf{Role} & You are a security expert well-versed in the legitimate behavior of crucial Windows processes. Now, you are qualified to assist in constructing behavior trees for legal processes. \\
\hline
\textbf{Goal} & 
Based on the current knowledge, please update the original \colorbox{SoftRed}{\{behavior\_tree\}}.
Please review the tree and determine if further refinement or subdivision is required for the current behaviors listed under the last branch. If, based on your expertise, no further details are needed for a specific behavior, please end the branch..
\\ 
\hline

\multirow{5}{*}{\textbf{Notes}} 
& 1. Each token has a cost, so keep your output concise and avoid irrelevant information. \\
& 2. Continuously review and analyze your actions to ensure you are performing to the best of your abilities. \\
& 3. Constructively self-criticize your big-picture behavior constantly. \\
& 4. The best would be to generate behaviors that can be verified within the system command. \\
\hline

\textbf{\colorbox{codeorange}{LLM Response}} \\
& \colorbox{lightgray} {2.1. Service Hosting: "Hosts multiple Windows services in the form of dynamic-link libraries (DLLs)} \\
& 2.1.1 DHCP Client Service": "Obtains an IP address and other related information from a DHCP server \\
& 2.1.1.1. IP Address Request": "Sends a request to the DHCP server to obtain an IP address \\
& 2.1.1.2. IP Address Renewal": "renew the lease on the current IP address \\
& .......\\
& 2.1.2. DNS Client Service": "Resolves and caches Domain Name System (DNS) names \\
& .... \\
& 2.1.7. Remote Procedure Call (RPC) Service": "Provides the endpoint mapper and other miscellaneous RPC services\\
\\
\hline
\end{tabularx}

\clearpage

\subsubsection{Commands Generation Prompt}
\label{prompt-commands}

\begin{tabularx}{\textwidth}{|c|X|}
\hline
\multicolumn{2}{|c|}{\textbf{Input:} \colorbox{SoftRed}{\{behavior\_tree\}}for\colorbox{codegreen}{\{process\_name\}}} \\
\multicolumn{2}{|c|}{\textbf{Output:} \colorbox{LightPeach}{\{commands\}}for\colorbox{codegreen}{\{process\_name\}}} \\
\hline
\textbf{Role} & You are an expert in cybersecurity, proficient in understanding the legitimate behaviors of crucial Windows processes. You specialize in generating corresponding commands based on the behavior tree of the current process. \\
\hline
\textbf{Goal} &  \\
& You are a cybersecurity specialist well-versed in the legitimate behaviors of essential Windows processes. \\
& You will receive a current behavior tree that consists of various behaviors and sub-behaviors \\
& Each sub-behavior is followed by parentheses indicating whether it has been translated into a corresponding command. \\
& There are two methods for validation: \\
& 1 If a system command can be directly generated from the behavior, create the specific command and append it to the behavior tree like so: (Command: xxx). \\
& 2 If a system command cannot be generated, provide a corresponding recommendation such as rebooting the system, and append it to the behavior tree like so: (Suggestion: Suggestion) \\

\\ 
\hline
\multirow{5}{*}{\textbf{Notes}} 
& 1. Each token has a cost, so keep your output concise and avoid irrelevant information. \\
& 2. Continuously review and analyze your actions to ensure you are performing to the best of your abilities. \\
& 3. Constructively self-criticize your big-picture behavior constantly. \\
& 4. Ideally, behaviors should be generated that can be confirmed through system commands. \\
\hline
\textbf{\colorbox{codeorange}{LLM Response}} \\
& ...... \\
& C:/Windows/system32/svchost.exe -k appmodel -s StateRepository \\
& C:/Windows/system32/svchost.exe -k dcomlaunch -s LSM \\
& C:/Windows/system32/svchost.exe -k localService -s w32Time \\
& Get-DhcpServer4Scope -ScopeID \\
& Get-DhcpServerv4Lease IPAddress\\
& ...... \\
\\
\hline
\end{tabularx}

\clearpage

\subsubsection{Constraints Generation and Explanation}
\label{prompt-cons-explain}

\begin{tabularx}{\textwidth}{|c|X|}
\hline
\multicolumn{2}{|c|}{\textbf{Input:}  \colorbox{codegreen}{\{process\_behavior\_description\}}} \\
\multicolumn{2}{|c|}{\textbf{Output:} \colorbox{LightPeach}{\{explanation\}}} \\
\hline
\textbf{Role} & You are an expert in cybersecurity, proficient in understanding the legitimate behaviors of crucial Windows processes. You specialize in generating corresponding commands based on the behavior tree of the current process. \\
\hline
\textbf{Goal} &  \\
&  You are a seasoned security expert with a deep understanding of Windows system processes. \\
& You will be presented with a \{process\_behavior\_description\}. \\
& This description can convert a corresponding command that, when executed on a system, generates a log. \\
& Analyze each log entry with a specific focus on: \\
&   1 DLL meanings \\
&    2 Registry key values meanings \\
&   3 File names meanings  \\
&    4 Process names meanings  \\
& Relevance Levels: \\
& 1 \{Essential\}   \\
& 2 \{Strongly\_Related\}   \\
& 3 \{Possibly\_Related\}  \\
& 4 \{Unrelated\}   \\
& For each log entry, categorize its relevance levels and provide an explanation for categories \{Essential\} 
\\ 
\hline
\multirow{5}{*}{\textbf{Example}} 
& Log Entry: RegQueryValue,hklm/system/currentcontrolset/control/scmconfig/*.\\
& [RegQueryValue,hklm/system/currentcontrolset/control/scmconfig/*][Essential] [Explanation: The process frequently queries the Windows Registry to retrieve configuration or settings. The registry key path hklm/system/currentcontrolset/control/scmconfig* is associated with configurations related to the Service Control Manager (SCM).] \\
& 3. Constructively self-criticize your big-picture behavior constantly. \\
& 4. Ideally, behaviors should be generated that can be confirmed through system commands. \\
\\
\hline
\end{tabularx}

\subsubsection{Temporal Constraint Validation}
\label{prompt-init-temporal}

\begin{tabularx}{\textwidth}{|c|X|}
\hline
\multicolumn{2}{|c|}{\textbf{Input:} \colorbox{SoftRed}{\{log\_Sequence\}}for\colorbox{codegreen}{\{process\_name\}}} \\
\multicolumn{2}{|c|}{\textbf{Output:} \colorbox{LightPeach}{\{These two logs must appear in this specific order.\}}for\colorbox{codegreen}{\{process\_name\}}} \\
\hline
\textbf{Role} & You are a security expert and will receive logs from a certain process. Using your expertise, determine if there is a sequential dependency between these logs. If such a dependency exists, please identify it and explain the rationale behind this relationship. \\
\hline
\textbf{Notes} &  \\
& Analyze each log entry individually and pinpoint logs that must appear in a specific order. Refrain from providing superfluous explanations. \\
\\ 
\hline
\textbf{\colorbox{codeorange}{LLM Response}} \\
& [lsass.exe,Load,c:/windows/system32/lsasrv.dll, then Load, c:/windows/system32/samsrv.dll] \\
& lsasrv.dll  serves as a core component of the LSA, and often needs to interact with SAM and samsrv.dll provides the essential interfaces and functions for interacting with the SAM database.
\\
\hline
\end{tabularx}


\clearpage
\subsubsection{Cross-session Validation}
\label{prompt-cross-validation}


\begin{tabularx}{\textwidth}{|c|X|}
\hline
\multicolumn{2}{|c|}{\textbf{Input:} \colorbox{SoftRed}{\{process\_behavior\}}for\colorbox{codegreen}{\{process\_name\}}} \\
\multicolumn{2}{|c|}{\textbf{Output:} \colorbox{SoftRed}{\{yes\}}or\colorbox{codegreen}{\{No\}}} \\
\hline
\textbf{Role} & You are Agent1, a cybersecurity specialist with expertise in legitimate behaviors of key Windows processes. There are a total of 3 agents. \\
\hline
\textbf{Goal} & 
\\
& We aim to verify the legitimacy of a program's behavior.\\
&  To achieve this, different LLM sessions will debate the behaviors, ensuring their accuracy and consistency. \\
& For the provided behaviors, explain their legitimacy. Why do you believe these behaviors are inherent to the process in question? \\
& You will also receive feedback from other LLM sessions. The format is: \\
& Agent 1: "This behavior is inherent to process XX because..." \\
& Agent 2: "This behavior..." \\
& ...... \\
\\ 
\hline

\multirow{5}{*}{\textbf{Other Agent}} 
& 1. You are Agent2, a cybersecurity specialist with expertise in legitimate behaviors of key Windows processes. There are a total of 3 agents.. \\
& 2. You are Agent3, a cybersecurity specialist with expertise in legitimate behaviors of key Windows processes. There are a total of 3 agents. \\
\hline

\multirow{5}{*}{\textbf{Notes}} 
& 1. Keep your answer as concise as possible, just yes or no, and follow it up with the reason why \\
& 2. After engaging in several rounds of debate, if you are confident in your assessment, please respond with "yes". \\
\hline

\textbf{\colorbox{codeorange}{LLM Response}} \\
& \colorbox{lightgray} {xxx} \\
\\
\hline
\end{tabularx}






%%%%%%%%%%%%%%%%%%%%%%%%%%%%%%%%%%%%%%%%%%%%%%%%%%%%%%%%%%%%%%%%%%%%%%%%%%%%%%%%
\end{document}
%%%%%%%%%%%%%%%%%%%%%%%%%%%%%%%%%%%%%%%%%%%%%%%%%%%%%%%%%%%%%%%%%%%%%%%%%%%%%%%%

%%  LocalWords:  endnotes includegraphics fread ptr nobj noindent
%%  LocalWords:  pdflatex acks
