%-------------------------------------------------------------------------------
\section{Introduction}
%-------------------------------------------------------------------------------
Advanced Persistent Threats (APTs) have emerged as a formidable adversary in the cyber landscape. Their stealthy, persistent nature combined with the extensive damage they inflict makes them particularly challenging to detect and counter. In light of this, we embarked on an exhaustive study of current commercial Endpoint Detection and Response (EDR) products \cite{karantzas2021empirical} and provenance graph-based solutions \cite{cheng2023kairos,alsaheel2021atlas,han2020unicorn,inam2022sok,han2021sigl}. Our deep dive into these systems allowed us to identify critical gaps and challenges. Specifically, we distilled the problems facing current solutions into four key dimensions:

\begin{itemize}
    \item \textbf{Insufficient information within Log Data}: : Effective intrusion detection hinges on the richness and comprehensiveness of audit logs. Recent research \cite{gandhi2023rethinking} underscores a concerning deficiency: current logging systems capture insufficient information for reliable attack detection. Solely relying on these raw logs may result in a low attack detection coverage. To enhance this, there's a pressing need to incorporate additional knowledge, such as specific meanings of entities within logs. By enriching our logs with more information and knowledge, we stand a better chance of detecting a wider range of attacks, especially the stealthy behaviors often employed in APT attacks.
    \item \textbf{Attack Agnosticity}: 
    As zero-day vulnerabilities (malware or flaws unknown to security analysts) become increasingly prevalent, particularly in APT attacks, detection systems that aren't reliant on any pre-existing attack signatures or signals offer greater versatility in detecting new threats. Consequently, the most effective approach for detecting such zero-day attacks is through anomaly detection.
    \item \textbf{Evolving Threat}: Advanced Persistent Threats (APT) represent dynamic, sustained, and sophisticated threats. Professional attacker groups continuously innovate, targeting an ever-expanding array of system processes and adapting to the latest defensive measures. This evolving landscape necessitates detection methods that can swiftly identify and adapt to emerging attack paradigms.
    \item \textbf{Transparency and Practicality}: The value of detection alerts lies in their clarity. For security personnel, a clear understanding of the 'why' and 'how' behind detections can drive quicker and more effective interventions. Furthermore, the practicality of a detection solution is gauged by its compatibility with established commercial security platforms. Seamless integration ensures a smooth transition from research prototypes to real-world applications.
\end{itemize}

While data provenance-based threat detection has emerged as a promising approach against the covert Advanced Persistent Threats (APTs), current methodologies exhibit specific shortcomings, failing to satisfy all four dimensions concurrently.
Commercial products designed to detect APT attacks, along with \textbf{misuse-based} APT research \cite{milajerdi2019holmes,milajerdi2019poirot,hossain2020combating}, predominantly hinge on rules defined by security experts. While these misuse-oriented strategies excel at identifying known threats, they lack flexibility when confronted with novel, undefined threats. Consequently, they are constrained by predefined security policies, struggling to swiftly track evolving attacks.
On the other hand, while \textbf{anomaly-based} techniques \cite{wang2022threatrace,han2020unicorn,wang2020you} are adept at spotting deviations, their outcomes often lack transparency and pose challenges for integration into commercial products. Within this realm, graph \cite{manzoor2016fast,han2020unicorn,li2021hierarchical,yang2023prographer,cheng2023kairos} and path-based methods \cite{wang2020you,alsaheel2021atlas} tend to detect only those attacks that leave discernible traces on the graph or path, yielding results that are too macroscopic. Node-based methods provide finer granularity but can sometimes oversimplify intricate attack behaviors. Knowledge graph embedding techniques \cite{zeng2021watson,zengy2022shadewatcher}, though potent in behavior extraction, often miss crucial attack semantics. 
\textbf{Statistical-based} techniques \cite{liu2018towards,hassan2019nodoze,hassan2020we}, although intuitive, grapple with false positives, making it tough to differentiate genuinely anomalous behaviors from benign novelties. 
In sum, while each technique offers nuanced advantages, a comprehensive method that addresses all four dimensions and seamlessly tackles every facet of APT detection remains an open challenge.


The primary impediments to addressing the current challenges in APT detection can be attributed to the inherently knowledge-intensive nature of the cybersecurity domain. When faced with audit logs that present insufficient information, there's an acute need to augment them with supplementary domain-specific knowledge. This supplemental knowledge becomes even more pivotal when attacks evolve or present themselves in stealthy manners. Moreover, a well-rounded domain understanding also serves as a foundation for elucidating detection results, enhancing transparency for security analysts. Thus, the key to overcoming these challenges lies in the ability to swiftly and effectively harness extensive cybersecurity knowledge.
Fortuitously, recent advancements in Large Language Models (LLMs) are showcasing remarkable prowess. These models exhibit a profound ability to understand human-like text intricacies and have demonstrated efficacy across a plethora of domains, especially in scenarios necessitating rapid knowledge extraction. This raises a pivotal query: \textbf{Can LLMs be effectively leveraged in cybersecurity to transform detection methods for stealthy and evolving APT attacks?}

Motivated by this pivotal question, we are poised to address these dimensions, crafting a solution that melds the knowledge extraction prowess of Large Language Models (LLM) with the structural acumen of provenance graphs, aiming for a holistic, adaptive, and practical APT detection. It's imperative to recognize that the domain of knowledge in cybersecurity encompasses both benign system behaviors and potential attack patterns. Given our aspiration to address the challenges posed by unknown attack vectors, our focus gravitates towards the autonomous extraction of knowledge pertaining to critical system processes and their typical behaviors. In summary, we hope our approach taps into the LLM's ability to extract knowledge, allowing us to build a detailed profile of system processes, emphasizing its essential constraints. The constraints can be seamlessly incorporated into intrusion detection rules, enhancing our ability to detect malicious activities.

While leveraging LLMs to construct process profiles, we conducted experiments to gauge their strengths and limitations. LLMs, with their vast knowledge base, showed a profound understanding of process behaviors and could explaining their outputs. Yet, we also encountered several notable challenges:

\begin{itemize}
    \item Context Dependency: LLMs often require a rich context to provide accurate results. Absent this, their responses can be generic or imprecise.
    \item Memory Limitations: LLMs tend to emphasize recent interactions in a conversation history, often overlooking or forgetting previous details.
    \item Complexity Handling: When faced with multifaceted questions or scenarios, LLMs might struggle to produce comprehensive solutions.
    \item Hallucinations: There's a risk that the model might generate plausible-sounding yet baseless or non-existent information.
\end{itemize}

While LLMs bring a wealth of potential, directly applying them to craft program profiles has its challenges. In response, we developed ProfileGuard-LLM, a method that harnesses LLMs for the automated construction of vital system process profiles. We decomposed the comprehensive profiling task into three manageable sub-tasks:
\begin{itemize}
    \item Behavior Tree Construction: For each process, we aimed to capture a comprehensive range of actions, creating a behavior tree. The LLM enriched this tree using a "self-ask" approach, continuously expanding its knowledge. When faced with memory constraints due to the breadth of behaviors, we used an in-memory database caching strategy. This tree also played a pivotal role in furnishing the LLM with contextual depth.
    \item Command Execution: With insights gleaned from the behavior tree, the LLM scripted and executed commands for behaviors that could be actioned upon. Those behaviors not directly translatable were catalogued, both for validation to counteract potential LLM misinterpretations and to generate richer contextual information.
    \item Constraint Extraction: For complex temporal behavioral constraints, we employed the prefixspan sequence mining algorithm to unearth deeper relationships between logs. With the aid of the LLM's knowledge, we then obtained interpretations for each identified constraint.
\end{itemize}
Throughout this procedure, validating the LLM's outputs for accuracy was paramount. Beyond real-world execution validation, the entire process incorporated multiple sessions for cross-validation purposes.
In essence, our methodology offers a fully automated approach, enabling swift construction of new program profiles. At its core, our technique embodies a mechanism where the LLM engages in self-criticism, ensures self-consistency, and continually evolves, making it a dynamic, self-reflective, and adaptive system for automatic process profiling.

To rigorously evaluate ProCon's efficacy, we meticulously crafted our own Advanced Persistent Threat (APT) scenarios, drawing from a wide range of existing APT documentation and methodologies used in known cyberattacks. By simulating these advanced attacks, we ensured a testing environment that truly represents the challenges a real-world system might face. In comparing our solution against contemporary baselines, ProCon consistently showcased its prowess, delivering superior detection rates and a broader scope of threat identification. This solid performance, grounded in realistic and diverse attack simulations, underscores ProCon's potential to be a game-changer in the realm of cybersecurity, providing organizations with a heightened defense mechanism against the continually evolving landscape of APTs.


