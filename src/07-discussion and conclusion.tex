
\section{Discussion}

\textbf{Some attacks do not violate process constraints}: There are some attacks that do not violate process constraints. For example, attackers can launch sophisticated attacks by using common tools such as rundll32.exe and powershell.exe in a manner that does not violate process constraints. However, LLMs still hold the potential to solve this problem. From a blacklist perspective, LLMs can explore more logical vulnerabilities. From a whitelist perspective, LLMs can establish behavioral constraints at the logical behavior level.

\textbf{Issues with Commands Generated by LLMs}: There are issues with the commands generated by LLMs. Some commands cannot be executed in the system, and some are very difficult to run because they involve system processes. Also, it is challenging to determine whether some commands have been executed successfully.

\textbf{Incomplete Knowledge in LLMs}: LLMs lack comprehensive knowledge. A lack of process information in LLMs can lead to detection failures. Using external security information to supplement missing data could be one solution.

\textbf{Hallucination Problems in LLMs}: Although the accuracy of knowledge can be improved by using real logs and cross-session for validation, LLMs are difficult to completely eliminate illusions.

\section{Conclusion}