\section{Related work}

% \paragraph{Stealthy APT Attacks}
% Advanced Persistent Threats (APTs) are sophisticated, targeted, evolving, and steathy cyberattacks launched by specialized groups against high-risk entities, such as nuclear power plants, banks, and governments. APTs are using stealthy tactics such as name obfuscation\cite{cybereason2023,checkpoint2021,intrinsec}, process injection\cite{checkpoint2020,fsecure2019}, process hollow\cite{secureworks}, side-loading DLLs\cite{mitre_g0022,mitre_g0016}, and manipulating legitimate system utilities in order to bypass traditional security defenses due to the dynamic cyber threat landscape\cite{barr2021survivalism}. An example is Process injection, a popular technique for enabling malicious code to run within another process's address space.  Additionally, attackers use tools like \textit{rundll32.exe} to execute benign binaries and \textit{powershell.exe} to launch script-driven attacks\cite{li2019effective}. By exploiting these widely recognized and trusted tools, adversaries can discreetly penetrate and control their target systems.

% \paragraph{Provenance Graph-Based APT Detection}
% Chenyan  usenix chair
% Threat Detection and Investigation with System-level Provenance Graphs: A Survey
% CONAN: A Practical Real-time APT Detection System with High Accuracy and Efficiency
% Effective and Light-Weight Deobfuscation and Semantic-Aware Attack Detection for PowerShell Scripts
% AttacKG: Constructing Technique Knowledge Graph from Cyber Threat Intelligence Reports
% RATScope: Recording and Reconstructing Missing RAT Attacks for Forensic Analysis with Semantics on Windows
% CONAN: A Practical Real-time APT Detection System with High Accuracy and Efficiency

% PC member
% Kangkook Jee
% You Are What You Do: Hunting Stealthy Malware via Data Provenance Analysis

% Ivan Martinovic
% Survivalism: Systematic Analysis of Windows Malware Living-Off-The-Land

% Z. Berkay Celik
% ATLAS: A Sequence-based Learning Approach for Attack Investigation

% Xiapu Luo
% D IST D ET: A Cost-Effective Distributed Cyber Threat Detection System

% Adil Ahmad
% Rethinking System Audit Architectures for High Event Coverage and Synchronous Log Availability
In this section, we discuss the related work to anomaly-based detection methods. Given the expansive body of literature regarding this topic, we will only discuss the most prominent ones therein, i.e., provenance graph-based approaches, which encapsulate two main detection strategies: learning-based detection and rule-based detection.

% Provenance graph-based methods are becoming increasingly popular in detecting stealthy APT attacks, due to their ability to identify a variety of host-based threats. There are three main detection methods based on Provenance graphs: learning-based detection, and rule-based detection.

\paragraph{Learning-based Detection}
For the learning-based detection methods, we broadly classify the existing approaches into four categories based on the granularity of anomaly detection: (1) graph-based, (2) graph-based, (3) path-based, (4) node-based, and (5) knowledge graph embedding-based.

The first class of detection techniques train models on benign behaviors, identifying deviations as potential cyber-attacks \yd{Add refs}. Although these methods can detect threats with impressive accuracy by integrating audit record semantics into threat analysis, existing learning solutions often fail to provide insightful and explicable results, compromising the practical utility.
% 

\noindent
{\bf Graph-based Methods.} Graph-based anomaly detection requires extensive training datasets and is computationally intensive. The researchers have embedded provenance graphs into vector space and developed tools such as StreamSpot \cite{manzoor2016fast} and Unicorn \cite{han2020unicorn} to analyze information flow graphs. Despite Unicorn's superior performance due to its thorough graph analysis, both methods face challenges in detecting stealthy threats due to graph kernel limitations. In the same way, IPG \cite{li2021hierarchical} and ProGrapher \cite{yang2023prographer} use graph-level approaches, but struggle with similar issues as StreamSpot and Unicorn. Despite the fact that many systems attempt to use data provenance for threat detection, methods that combine scope and timeliness are needed. The remarkable capabilities of KAIROS \cite{cheng2023kairos} make it an excellent competitor.

\noindent
{\bf Path-based Methods.} By extracting causal paths from the provenance graph, these models leverage existing learning methodologies. ProvDetector \cite{wang2020you} analyzes the provenance graph for malware detection by converting paths into embedded forms and utilizing the Local Outlier Factor method. However, due to the diversity of host-based threats, relying solely on paths from the provenance graph is not sufficient. ATLAS \cite{alsaheel2021atlas}uses sequence models to distinguish attack patterns, recognizing that explicit and abstract strategies may be analogous, regardless of vulnerabilities exploited and payloads executed.

\noindent
{\bf Node-based Methods.} To detect stealthy abnormal behavior without knowing prior attack patterns, ThreaTrace\cite{wang2022threatrace} employs a GraphSAGE-based framework that learns every benign node's role in a system data provenance graph. To address data imbalance and improve detection, ThreaTrace provides a multi-model framework to learn different types of benign nodes.

\noindent
{\bf Knowledge Graph Embedding-based Methods.} Tools like Shadewatcher\cite{zengy2022shadewatcher} and Watson\cite{zeng2021watson} both use knowledge graph embedding to represent the semantics of individual nodes and edges. This is a promising approach. The tactics used by attackers can, however, be varied and complex. An attacker who creates a malicious node that closely mimics a genuine node makes it difficult to identify it. Process hollowing attacks, for instance, use semantic information nearly identical to that of the functionality used to fill the hollowed node, making detection difficult. As well, it may be difficult to accurately represent a single process with a single vector because a single process may encompass multiple functionalities.


% \paragraph{Statistics-based Approaches.} Recent research suggests that security incidents in attack campaigns typically manifest as uncommon system activities \cite{liu2018towards,hassan2019nodoze,hassan2020we}. According to these studies, audit records are analyzed based on their historical frequency. Although this approach is straightforward and effective, it often leads to false positives. It is possible for an alert to be triggered by an activity simply because it hasn't been observed before,  even if it is a benign update to the process status. Its primary limitation is its inability to distinguish between genuinely unusual records and new but semantically normal activities.

\paragraph{Rules-based Approaches}
A misuse-based detector detects cyber threats by comparing audit records with an attack semantics knowledge base.
Developing security policies, however, is time-consuming and requires domain expertise, despite such detection maintaining a low false-positive rate.
Holmes \cite{milajerdi2019holmes} uses prior definitions of exploits in a provenance graph based on existing TTPs (Tactics, Techniques, and Procedures).
Based on the expertise of cyber threat reports, Poirot\
cite{milajerdi2019poirot} focuses on correlated indicators and constructing attack graphs.
A Morse command called \cite{hossain2020combating} is used to propagate the integrity and confidentiality tags for six million system entities.
In spite of this, misuse-based methods have difficulty detecting unknown threats that do not meet established TTPs and reports.


%\subsection{Large Language Models}
%LLMs\cite{radford2018improving,radford2018improving,ouyang2022training}, such as OpenAI's GPT series, have revolutionized natural language understanding. Since these models have been trained extensively on corpora, they have a wide range of knowledge and reasoning capabilities, which makes them suitable for a variety of tasks associated with natural language processing. Using a simple natural language prompt, the LLMs can execute designated tasks without any specific retraining. Using the Transformer\cite{vaswani2017attention} model, LLMs interpret input prompts and generate corresponding answers, where multi-self-attention and feed-forward layers work together to interpret context.
%Meanwhile, LLMs are used to construct knowledge graphs for automated knowledge extraction\cite{zhu2023llms,pan2023unifying}, combining the capabilities of large language models with the accuracy of knowledge graphs.
%
%In the field of cybersecurity, LLMs are gaining a considerable amount of attention. The abilities of LLMs to understand, infer, and generate text have a positive impact on computer science and cybersecurity. Its effectiveness has been proven in areas such as code analysis\cite{sun2023gpt,kang2023large}, code generation\cite{liu2023improving}, program repair\cite{wei2023copiloting}, vulnerability description mappings\cite{liu-etal-2023-end}, security test\cite{zhang2023well}, fuzzing test\cite{zhao2023understanding} and penetration Testing\cite{deng2023pentestgpt}. Fine-tuning LLMs are also utilized in the field of cybersecurity, adapting them to suit various tasks within the domain.
%With their superior knowledge extraction capabilities, LLMs may be able to detect advanced threats because of their versatility and human-like interaction.





